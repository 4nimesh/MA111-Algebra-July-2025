\documentclass{article}
\usepackage{graphicx} % Required for inserting images
\usepackage{amsmath,amssymb,amsthm, mathrsfs}
\usepackage[colorlinks=true,linkcolor=blue,citecolor=blue,pdfpagelabels=false]{hyperref}

\title{Algebra Tutorial-I}
\begin{document}
% \maketitle
\section*{Equivalence relation}
% \subsection
An \emph{equivalence relation} on a set \( S \) is a subset \( R \subseteq S \times S \) satisfying the \emph{reflexivity}, \emph{symmetry} and \emph{transitivity} properties.
We write \( a \sim b \) to indicate that \( (a, b) \in R \), and say that \( \sim \) is the equivalence relation.
\subsection*{Problems}
\begin{enumerate}
    \item Let \( A = \{0, 1, 2, 3\} \) and define a relation \( R \) on \( A \) by
\[
R = \{(0,0),\ (0,1),\ (0,3),\ (1,0),\ (1,1),\ (2,2),\ (3,0),\ (3,3)\}.
\] Is it an equivalence relation? If so, what are the equivalence classes?
\item For $a,b\in \mathbb Z$, let $a\sim b$ iff $a-b$ is even. Show that $\sim$ is an equivalence relation. {What are the equivalence classes}?
\item Let \( \mathbb{N} \) be the set of natural numbers, and define the relation \( R \subseteq \mathbb{N} \times \mathbb{N} \) by
$R = \{ (x, y) \in \mathbb{N} \times \mathbb{N} \mid 2x + y = 10 \}.$ Show that the relation is neither reflexive, nor symmetric, nor transitive.
\item Define a partition of a set. 

Show that every equivalence relation on a set $S$ yields a partition of the set, and vice versa.
\item Let $n$ be a fixed positive integer. Define a relation on $\mathbb Z$ by \[a\sim b \quad\text{ iff }\quad n\vert(b-a).\] Show that the relation is an equivalent one and determine the equivalence classes.

The set of equivalence classes of this relation is denoted by $\mathbb Z/n\mathbb Z.$

\item Let us define a relation on $\mathbb R^n \setminus\{(0,0,\cdots, 0) \}$ by
\[(a_1,a_2,\cdots,a_n)\sim (b_1,b_2,\cdots,b_n) \quad\text{ iff }\quad a_i=\lambda b_i,\quad \text{($i=1,2,\cdots,n$})\] for some nonzero $\lambda\in \mathbb R.$ The equivalence classes are the lines through the origin and the set of all equivalence classes is called the real projective space $\mathbb{RP}^n.$

\item The `similarity' relation on $M_n(\mathbb R)$ is defined by \[A\sim B \quad\text{ iff }\quad B=CAC^{-1}\] for some invertible $C\in M_n(\mathbb R).$ Show that the relation is an equivalence relation.

What about the `congruence' relation defined by \[A\sim B \quad\text{ iff }\quad B=CAC^{\mathrm T}?\]
\end{enumerate}











\end{document}