\documentclass{article}
\usepackage{graphicx} % Required for inserting images
\usepackage{amsmath,amssymb,amsthm, mathrsfs}
\usepackage[colorlinks=true,linkcolor=blue,citecolor=blue,pdfpagelabels=false]{hyperref}

\title{Algebra Tutorial-I}
\begin{document}
% \maketitle
\section*{Equivalence relation}
% \subsection
Let \( S \) be a set. An \emph{equivalence relation} on \( S \) is a subset \( R \subseteq S \times S \) satisfying the following three properties:

\begin{itemize}
  \item[(i)] {Reflexivity:} For all \( a \in S \), \( (a, a) \in R \).
  \item[(ii)] {Symmetry:} For all \( a, b \in S \), if \( (a, b) \in R \), then \( (b, a) \in R \).
  \item[(iii)] {Transitivity:} For all \( a, b, c \in S \), if \( (a, b) \in R \) and \( (b, c) \in R \), then \( (a, c) \in R \).
\end{itemize}

We write \( a \sim b \) to indicate that \( (a, b) \in R \), and say that \( \sim \) is the equivalence relation.
\subsection*{Problems}
\begin{enumerate}
    \item Let \( A = \{0, 1, 2, 3\} \) and define a relation \( R \) on \( A \) by
\[
R = \{(0,0),\ (0,1),\ (0,3),\ (1,0),\ (1,1),\ (2,2),\ (3,0),\ (3,3)\}.
\] Is it an equivalence relation? If so, what are the equivalence classes?
\item For $a,b\in \mathbb Z$, let $a\sim b$ iff $a-b$ is even. Show that $\sim$ is an equivalence relation. {What are the equivalence classes}?
\item Let \( \mathbb{N} \) be the set of natural numbers, and define the relation \( R \subseteq \mathbb{N} \times \mathbb{N} \) by
$R = \{ (x, y) \in \mathbb{N} \times \mathbb{N} \mid 2x + y = 10 \}.$ Show that the relation is neither reflexive, nor symmetric, nor transitive.

% We now check whether \( R \) is reflexive, symmetric, and transitive.
% \begin{itemize}
%   \item \textbf{Reflexive:} A relation \( R \) is reflexive if \( (x,x) \in R \) for all \( x \in \mathbb{N} \). In this case,
%   \[
%   2x + x = 3x = 10 \Rightarrow x = \frac{10}{3},
%   \]
%   which is not a natural number. So \( (x,x) \notin R \) for any \( x \in \mathbb{N} \).\\
%   \textit{Therefore, \( R \) is not reflexive.}

%   \item \textbf{Symmetric:} A relation \( R \) is symmetric if \( (x,y) \in R \Rightarrow (y,x) \in R \). Suppose \( (x, y) \in R \), so \( 2x + y = 10 \). For symmetry, we need \( 2y + x = 10 \). But in general, \( 2x + y = 10 \nRightarrow 2y + x = 10 \). For example:
%   \[
%   (x,y) = (3,4) \Rightarrow 2(3) + 4 = 10 \text{ is in } R, \text{ but } 2(4) + 3 = 11 \neq 10,
%   \]
%   so \( (4,3) \notin R \).\\
%   \textit{Therefore, \( R \) is not symmetric.}

%   \item \textbf{Transitive:} A relation \( R \) is transitive if for all \( x, y, z \in \mathbb{N} \),
%   \[
%   (x, y) \in R \text{ and } (y, z) \in R \Rightarrow (x, z) \in R.
%   \]
%   Suppose \( (x, y) \in R \Rightarrow 2x + y = 10 \), and \( (y, z) \in R \Rightarrow 2y + z = 10 \). We want to check whether \( 2x + z = 10 \). Adding the two equations:
%   \[
%   (2x + y) + (2y + z) = 20 \Rightarrow 2x + 3y + z = 20,
%   \]
%   which does not imply \( 2x + z = 10 \) unless \( 3y = 0 \), which is impossible for natural numbers \( y \).\\
%   \textit{Therefore, \( R \) is not transitive.}
% \end{itemize}

% \textbf{Conclusion:} The relation \( R = \{(x, y) \in \mathbb{N} \times \mathbb{N} \mid 2x + y = 10\} \) is neither reflexive, nor symmetric, nor transitive.
\item Define a partition of a set. 

Show that every equivalence relation on a set $S$ yields a partition of the set, and vice versa.
\item Let $n$ be a fixed positive integer. Define a relation on $\mathbb Z$ by \[a\sim b \quad\text{ iff }\quad n\vert(b-a).\] Show that the relation is an equivalent one and determine the equivalence classes.
\end{enumerate}











\end{document}